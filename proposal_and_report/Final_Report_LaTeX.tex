\documentclass{article}

% Language setting
% Replace `english' with e.g. `spanish' to change the document language
\usepackage[english]{babel}

% Set page size and margins
% Replace `letterpaper' with `a4paper' for UK/EU standard size
\usepackage[a4paper,top=2cm,bottom=2cm,left=3cm,right=3cm,marginparwidth=1.75cm]{geometry}
\usepackage{setspace}

% Useful packages
\usepackage{amsmath}
\usepackage{graphicx}
\usepackage[colorlinks=true, allcolors=blue]{hyperref}

\title{Visualization of World Happiness Report from 2015 to 2021}
\author{Ming Xian, Silvey Yu, Mingqian Zheng\\[\bigskipamount]
\{mx565,ly1164,mz2392\}@gmail.com}

\begin{document}
\maketitle


% \setstretch{1.2}
\section{Overview}

How happy are people around the world? What are the happiest and unhappiest countries? Is there a pattern in where they are located? What factors affect the level of happiness in that country? What’s the trend of so-called happiness scores throughout the time?

In this \href{https://github.com/SilvesterYu/DATS-SHU235-Information-Visualization-Final-Project}{project}, we answer the questions above by visualizing the \href{https://worldhappiness.report/ed/2020/#appendices-and-data}{World Happiness Report} data during 2015 to 2021, illustrating a global picture (Figure~\ref{fig:geomap}, Figure~\ref{fig:line}, Figure~\ref{fig:bar}) of individual happiness and a time trend for the countries involved. 

Here is a link to the data and our source code: \href{https://github.com/SilvesterYu/DATS-SHU235-Information-Visualization-Final-Project}{https://github.com/SilvesterYu/DATS-SHU235-Information-Visualization-Final-Project}.

\begin{figure}[!htb]
	\begin{center}
		\includegraphics[scale=0.3]{geomap.png}
	\end{center}
	\caption{The world map of happiness score for 2019}
	\label{fig:geomap}
\end{figure}

\begin{figure}[!htb]
\begin{minipage}[b]{.48\textwidth}
\centering
\includegraphics[width=1\textwidth]{line2.png}
\caption{The line chart of the continent's average happiness score}
\label{fig:line}
\end{minipage}
\hfill
\begin{minipage}[b]{.48\textwidth}
\centering
\includegraphics[width=1\textwidth]{bar2.png}
\caption{The bar chart of the top 20 countries' happiness score}
\label{fig:bar}
\end{minipage}
\end{figure}

\section{Data}
\subsection{Data Description}

We used a \href{https://www.kaggle.com/datasets/mathurinache/world-happiness-report}{Kaggle dataset} from the \href{https://worldhappiness.report/ed/2020/#appendices-and-data}{World Happiness Report} project, an on-going project that surveys people from all over the world for their happiness levels, and uses econometrics methods to analyse how much social, economic, and medical factors contributed to the happiness levels. The dataset covers most countries in the world from the year 2015-2022, and the 8 main variables are as below.

\begin{itemize}
    \item The “Happiness score” (i.e. Ladder score in 2020-2021) is a populated-weighted average score on a scale from 0 to 10. Its standard deviation is also recorded.
    \item There are 6 other columns indicating how many different socioeconomic factors contributed to the happiness score: GDP per capita, healthy life expectancy, social support, freedom to make life choices, generosity and corruption perception.
    \item There is a 7th factor “Dystopia”. It represents the lowest national averages for each key variable and is, along with residual error, used as a regression benchmark. Adding it with the other 6 factors will produce the “Happiness score”. 
\end{itemize}
\subsection{Data Preprocessing}
As the raw dataset is unorganized and unsuitable for our visualization analyses, we spent a considerable amount of time on data cleaning and merging. The steps are as below.
\begin{itemize}
    \item According to the definition, the addition of all the 7 component variables should equal the happiness score. During our verification process for the Kaggle dataset. The 2022 data failed to meet the criteria. So we excluded year 2022.
    \item Countries vary a little from year to year. To maintain consistent comparisons of happiness level across time, we only kept the countries that appear every year from 2015 to 2021, using the VLOOKUP function of Excel. The number of countries is 135. The happiness rank for each country was manually re-assigned after excluding some countries.
    \item The variables and countries are named differently each year. We transformed all of them into a uniform standard, which makes it convenient for visualization.
    \item For countries that miss the "region" variable, we match them according to the standard applied to other years, using the VLOOKUP function of Excel.
    \item For easier comparison, we transformed all variables to 4 decimal places.
\end{itemize}

\section{Goals and Tasks}
\textbf{Mainly introduced in a scenario-based application, with explanations using abstract visualization languages}\\

Sophie is an undergraduate student considering which countries to go to for education and work after graduation. She would like to live in a happy country where she can also feel happy–preferably a place that has freedom, good healthcare system, and stable economy. Sophie wants to have a quick overview of how happy each country is–whether or not people there feel happy, and what aspects of their country make them happy--and where these countries are located. When Sophie logs on the system, she will be able to choose a specific year of her interest using the slide bar. On our geomap, she can see at first glance the happiness level of all countries by different colors (saturation level). Hovering the mouse over the country, Sophie can see from the pie chart that which aspects–economy, healthcare, or freedom, etc., and how much they contributed to the happiness of people there. The pie chart also includes the country's happiness rank and score.

Looking to right hand side of the geomap, Sophie is able to see a line chart showing the average happiness score trend over the time period 2015-2021, where each line represents a continent. In this way, Sophie will develop a better sense of the relative position of her interested continent's happiness level and how it changed throughout the time. When Sophie hovers the mouse on each line, she can also easily understand where the continent locates as the corresponding continent on the geomap will be highlighted.

Additionally, the drop down menu below the slide bar enables Sophie to choose top xx (e.g. top 5, top 10) happiest countries in a certain year. The result will be shown through a bar chart, where x-axis is country and y-axis is happiness score. Moreover, the bar chart is be sorted in descending order for Sophie to have an intuitive impression. 

\section{Description of Visualization}
\begin{figure}[!htb]
	\begin{center}
		\includegraphics[scale=0.35]{linked.png}
	\end{center}
	\caption{The overview of the linked views}
	\label{fig:geomap2}
\end{figure}

\begin{figure}[!htb]
\begin{minipage}[b]{.55\textwidth}
\centering
\includegraphics[width=1\textwidth]{line_map.png}
\caption{Interaction between bar chart and geomap}
\label{fig:line}
\end{minipage}
\hfill
\begin{minipage}[b]{.40\textwidth}
\centering
\includegraphics[width=1\textwidth]{bar_map.png}
\caption{Interaction between line chart and geomap}
\label{fig:bar}
\end{minipage}
\end{figure}

A \textbf{line chart} of each region's happiness trend during 2015 and 2021 gives users a static overview of the global happiness trend. We use line mark with amplified points with hue channel. Each line with a distinctive color represents a region and the nodes along it refer to the average happiness score in each year.

A \textbf{world map/geomap} is filled with color indicating happiness level of each country at the year specified by the slide bar. We use the saturation channel to encode the happiness level where higher scores correspond to greater saturation.   

A \textbf{bar chart} shows the happiness scores of top X countries in descending order at a certain year collectively determined by the drop-down bar and slide bar.


\subsection{User Interaction}
\subsubsection{Selection}
The \textbf{slide bar} at the top left corner allows user to select a specific year of happiness scores used in the world map. The \textbf{drop-down bar} below is designed for the section of top 5, 10, etc. happiest countries, and it will make other countries half-opaque to emphasize the filtered countries.

\subsubsection{MouseOver}
% \begin{figure}[!htb]
% 	\begin{center}
% 		\includegraphics[scale=0.5]{pie.png}
% 	\end{center}
% 	\caption{The pie chart of the composition of China's happiness score for 2018}
% 	\label{fig:piechart}
% \end{figure} 
\begin{itemize}
    \item \textbf{Geomap:} When the mouse hovers over a country on the map, a \textbf{MouseOver} the country will be highlighted on the map, along with the corresponding region line in line chart and the bar in bar chart. A \textbf{tooltip} will also appear with (1) country name (2) country's happiness level (3) country's happiness rank (4) \textbf{pie chart} showing the composition of the happiness score at that year. Users can see what factors contributed most to the happiness level, and how much they each contributed. The pie chart is created using the \href{https://recharts.org/en-US/}{Recharts library}. 
    
    \item \noindent \textbf{Line chart:} When the mouse is over a point of a specific line (for example that of Eastern Asia), a \textbf{tooltip} will appear, next to the selected point, displaying the region name, happiness score of that point, and the year of that point. the corresponding region (in this case, countries in Eastern Asia) on the map will be highlighted and other regions will be half opaque. 

\item \noindent \textbf{Bar chart:} When the mouse hovers on a bar, the selected bar and the area of that country in the world map will be highlighted. A \textbf{tooltip} will appear next to the selected bar displaying the country name, rank and score.
\end{itemize}






\section{Reflection}
\subsection{Project Development}

Initially, we drafted three different projects based on data availability, operational feasibility and variety of visualization effects. We chose this topic because its usage scenarios are very diverse with various target audiences. Most importantly, we believe visualizing the happiness levels with both vertical and horizontal comparisons is of great significance. Nowadays, people have more options to choose where to live and attach more importance to mental well-being.

During the implementation stage, we adjusted our schedule timely as we found the original dataset is far from ready to use. We explored several matching and cleaning methods using Excel, Python and Stata. Whenever we completed a feature, we made reflections on its significance -- we want to make sure our project is useful and easy to understand. 

The coding part took a considerable amount of time. Development details are as follows:

\begin{itemize}
    \item  \textbf{\href{https://github.com/SilvesterYu/DATS-SHU235-Information-Visualization-Final-Project/blob/main/src/index.js}{Geomap}} has many linked interactions (double-way linking with both the line chart and the bar chart) and the highlighting of countries involves many \href{https://github.com/SilvesterYu/DATS-SHU235-Information-Visualization-Final-Project/blob/main/src/index.js}{layers of selection} (i.e., top X countries selection, region selection linked from line chart, single-country highlighting linked to bar chart, and only those in the top X countries be highlighted when mouse hovers over their region line). region name, country name and year combined is used as an id to locate the country elements. An onMouseUpdate() function is used to keep track of the mouse positions, so when the mouse hovers over a bar, the tooltip will appear right next to the mouse. The color scheme is purple, considering other colors like red or green might not be friendly to people with color vision deficiency. 
    
    \item \textbf{\href{https://github.com/SilvesterYu/DATS-SHU235-Information-Visualization-Final-Project/blob/main/src/lineChart.js}{Line chart}} is coded from scratch using React and d3. The first attempt was to code it using Recharts, however, recharts is very limited in what users can configure, so to implement selection of single points on the lines, mouse interaction with tooltip, and linked selection with the geomap, manually coding it out was the only option. An auto ID generation is used for all lines and points, enabling mouse events on each specific point on the lines. When mouse hovers over a point on a line, only this selected line gets highlighted, all others are half opaque. The idea is to traverse through all lines and points on the graph, and only show their opacity to full when they include the name of the region. There is a json containing the data points for the lind chart. All lines are plotted one by one traversing through the json data and allocating subarrays, and then generating all possible IDs. Also, because the average scores don't span the range from 1 to 10, they are mainly between 4.5 to 7.5, an auto scaling according to min, max values of the lines is used, so we are showing only the effective range (from 4.5 to 7.5), and thus we have less clutter, lines are more apart and the trend is more visible. A \href{https://jscolor.com/}{Javascript color selector} is used when selecting the colors for the lines.
    
    \item \textbf{\href{https://github.com/SilvesterYu/DATS-SHU235-Information-Visualization-Final-Project/blob/main/src/barChart.js}{Bar chart}} has a flexible length for the x axis according to the number of top X countries selected. The larger the value of X, the longer the x axis is. Elements are identified with an id that combines country name and year. 
    
    \item \textbf{\href{https://github.com/SilvesterYu/DATS-SHU235-Information-Visualization-Final-Project/blob/main/src/toolTip.js}{Tooltip}} contains a pie chart, so it is coded using the pie chart component in \href{https://recharts.org/en-US/api/PieChart}{Recharts} for it provides simple and beautiful solutions to our barchart. 
\end{itemize}

Details are in the \href{https://github.com/SilvesterYu/DATS-SHU235-Information-Visualization-Final-Project}{source code}. 


\subsection{Changes in Goals}
\begin{itemize}
    \item As explained in previous sections, our visualization samples only cover the year 2015-2021 without 2022 due to its data inaccuracy.
    \item We did not include a line for a specific country's happiness score to compare with its continent. This is because this feature of comparing the country with its continent does not provide useful information to the audience. A user can first decide a satisfactory continent based on the line chart, and narrow their options to see a country in that continent by hovering the mouse on the world map. If they want to see this country's score, they can simply see it by the tooltip of the world map.
    
    \item For the drop-down bar, we did not include the option of selecting the countries with the lowest world ranking as planned. We believe our audience is more interested in seeing the top happiest countries when he chooses to use our platform. If he really wants to know which countries have lower rankings, he can have a rough idea by looking at the shallowest areas in the world map. 
\end{itemize}

\end{document}